\newlength{\defaultpenwidth}
\setlength{\defaultpenwidth}{0.04\masterunit}

\definecolor{defaultpencolor}{RGB}{43, 43, 43}

\tikzset{every path/.style={
    draw=defaultpencolor,
}}

\tikzset{every node/.style={
    inner sep=0pt,
    outer sep=0pt,
    node distance=0,
    text=defaultpencolor,
}}

\tikzset{element/.style={
    font=\large,
    inner sep=0.1\masterunit,
}}

\tikzset{subelement/.style={
    font=\small,
    inner sep=0.1\masterunit,
}}

\tikzset{collection element line width/.style={
    line width=\defaultpenwidth,
}}

\tikzset{subcollection element line width/.style={
    line width=.75\defaultpenwidth,
}}

\tikzset{collection element/.style={
    collection element line width,
    draw,
    element,
    inner sep=0,
    minimum height=\masterunit,
    minimum width=\masterunit,
}}

\tikzset{collection/.style={
    column sep=-\defaultpenwidth,
    nodes={
        collection element,
    },
    outer sep=-0.5\defaultpenwidth,
    row sep=-\defaultpenwidth,
}}

\tikzset{subcollection/.style={
    collection,
    row sep=-0.75\defaultpenwidth,
    nodes={
        subcollection element line width,
        minimum width=.75\masterunit,
        minimum height=.75\masterunit,
        font=\small,
    },
}}

\tikzset{empty collection/.style={
    decorate,
    decoration={
        amplitude=0.025\masterunit,
        angle=45,
        border,
        segment length=0.05\masterunit,
    },
    draw,
    line width=.3\defaultpenwidth,
    minimum height=\masterunit,
    minimum width=0.3\masterunit,
    node contents={},
}}

\tikzset{tuple of collections/.style={
    collection,
    inner sep=.25\masterunit,
    left delimiter=(,
    right delimiter=),
}}

\tikzset{tuple separator/.style={
    draw=none,
    minimum width=0,
    text height=\masterunit,
    font=\Huge,
    node contents={\,,\,\,},
}}

\tikzset{some/.style={
    draw,
    line width=\defaultpenwidth,
    rectangle split,
    rectangle split parts=2,
    rectangle split draw splits=false,
    node contents={\footnotesize{\texttt{Some}} \nodepart{two} #1},
    minimum size=.75\masterunit,
}}

\tikzset{none/.style={
    draw,
    line width=\defaultpenwidth,
    node contents={\footnotesize{\texttt{None}}},
    minimum size=.75\masterunit,
}}

\tikzset{exception draw/.style={
    line width=.02\masterunit,
    decorate,
    decoration={
        zigzag,
        segment length=0.1\masterunit,
        amplitude=0.0125\masterunit,
    },
}}

\tikzset{throw ->/.style={
    exception draw,
    ->,
}}

\tikzset{exception/.style={
    draw,
    exception draw,
    inner sep=.1\masterunit,
    node contents=\texttt{#1},
    tips=false,
}}


\tikzset{subexception/.style={
    exception={#1},
    font=\footnotesize,
}}

\newlength\flowwidth
\setlength\flowwidth{.75\defaultpenwidth}

\tikzset{flow/.style={
    line width=\flowwidth,
}}

\tikzset{flow ->/.style={
    flow,
    -{Triangle[flex,fill=defaultpencolor, angle'=30]},
}}

\tikzset{<- flow/.style={
    flow,
    {Triangle[flex,fill=defaultpencolor, angle'=30]}-,
}}

\newlength\subflowwidth
\setlength\subflowwidth{.5\defaultpenwidth}

\tikzset{subflow/.style={
    line width=\subflowwidth,
}}

\tikzset{subflow ->/.style={
    subflow,
    -{Triangle[flex,fill=defaultpencolor, angle'=30]},
}}

\tikzset{<- subflow/.style={
    subflow,
    {Triangle[flex,fill=defaultpencolor, angle'=30]}-,
}}

\tikzset{bridge flow/.pic={
    \fill [white] circle [radius=.09\masterunit];
    % \draw [flow, -, color=gray, draw opacity=.25] (-.1\masterunit, 0) arc [start angle=180, end angle=360, x radius=.1\masterunit, y radius=.05\masterunit];
    \draw [flow, -] (-.1\masterunit, 0) arc [start angle=180, end angle=0, radius=.1\masterunit];
}}

\tikzset{jump/.pic={
    \begin{scope}[rotate=#1]
    \fill [white] circle [radius=.09\masterunit];
    \draw [line cap=rect] (-.1\masterunit, 0) arc [start angle=180, end angle=0, x radius=.1\masterunit, y radius=.075\masterunit];
    \draw [line cap=rect] (0, .1\masterunit) -- +(0, -.2\masterunit);
    \end{scope}
}}

\tikzset{solid dashed solid/.style={
    /utils/exec=\csname tikz@options\endcsname,
    decorate,
    decoration={
        show path construction,
        lineto code={
            \path [tips=false] coordinate (_a) at ($ (\tikzinputsegmentfirst)!.15!(\tikzinputsegmentlast) $);
            \path [tips=false] coordinate (_b) at ($ (\tikzinputsegmentfirst)!.85!(\tikzinputsegmentlast) $);
            \draw [tips=false, solid] (\tikzinputsegmentfirst) -- (_a);
            \draw [tips=false, dashed] (_a) -- (_b);
            \draw [solid] (_b) -- (\tikzinputsegmentlast);
        }
    },
}}

\tikzset{solid dashed/.style={
    /utils/exec=\csname tikz@options\endcsname,
    decorate,
    decoration={
        show path construction,
        lineto code={
            \path [tips=false] coordinate (_a) at ($ (\tikzinputsegmentfirst)!.4!(\tikzinputsegmentlast) $);
            \draw [tips=false, solid] (\tikzinputsegmentfirst) -- (_a);
            \draw [dashed] (_a) -- (\tikzinputsegmentlast);
        }
    },
}}

\tikzset{dashed solid/.style={
    /utils/exec=\csname tikz@options\endcsname,
    decorate,
    decoration={
        show path construction,
        lineto code={
            \path [tips=false] coordinate (_a) at ($ (\tikzinputsegmentfirst)!.6!(\tikzinputsegmentlast) $);
            \draw [tips=false, dashed] (\tikzinputsegmentfirst) -- (_a);
            \draw [solid] (_a) -- (\tikzinputsegmentlast);
        }
    },
}}


\tikzset{
    elements before/.style={
        /utils/exec=\csname tikz@options\endcsname,
        alias=this,
        draw=none,
        minimum width=#1\masterunit,
        node contents={},
        append after command={
            \pgfextra{
                \draw[dashed solid] (this.north west) -- (this.north east);
                \draw[dashed solid] (this.south west) -- (this.south east);
            }
        },
    },
    elements before/.default=.5,
}

\tikzset{
    elements between/.style={
        /utils/exec=\csname tikz@options\endcsname,
        alias=this,
        draw=none,
        minimum width=#1\masterunit,
        node contents={},
        append after command={
            \pgfextra{
                \draw[solid dashed solid] (this.north west) -- (this.north east);
                \draw[solid dashed solid] (this.south west) -- (this.south east);
            }
        },
    },
    elements between/.default=1.25,
}

\tikzset{
    elements after/.style={
        /utils/exec=\csname tikz@options\endcsname,
        alias=this,
        draw=none,
        minimum width=#1\masterunit,
        node contents={},
        append after command={
            \pgfextra{
                \draw[solid dashed] (this.north west) -- (this.north east);
                \draw[solid dashed] (this.south west) -- (this.south east);
            }
        },
    },
    elements after/.default=.5,
}

\tikzset{
    vertical elements before/.style={
        rotate=90,
        elements before=#1,
    }
}

\tikzset{
    vertical elements between/.style={
        rotate=90,
        elements between=#1,
    }
}

\tikzset{
    vertical elements after/.style={
        rotate=-90,
        elements after=#1,
    }
}

\tikzset{key to value/.style={
    collection element line width,
    decorate,
    decoration={
        show path construction,
        lineto code={
            /utils/exec=\csname tikz@options\endcsname,
            \fill (\tikzinputsegmentfirst) [white] circle [radius=.15\masterunit];
            \draw ($ (\tikzinputsegmentfirst) + (0, .15\masterunit) $) [draw, line cap=rect] arc [start angle=90, end angle=270, radius=.15\masterunit];

            \fill [defaultpencolor] (\tikzinputsegmentfirst) circle [radius=.05\masterunit];
            \draw (\tikzinputsegmentfirst) [flow, defaultpencolor, -Triangle] -- (\tikzinputsegmentlast);
        }
    }
}}

\tikzset{function/.style={
    shape=function,
    line width=.75\defaultpenwidth,
    minimum width=.75\masterunit,
    minimum height=.75\masterunit,
    /function/io width=.3\masterunit,
    /function/io narrow width=.2\masterunit,
    /function/io height=.15\masterunit,
    /function/io sep=.2\masterunit,
}}

\tikzset{subfunction/.style={
    shape=function,
    inner xsep=.15\masterunit,
    inner ysep=.2\masterunit,
    line width=.5\defaultpenwidth,
    /function/io width=.2\masterunit,
    /function/io narrow width=.133\masterunit,
    /function/io height=.1\masterunit,
    /function/io sep=.1\masterunit,
}}

\tikzset{horizontal subfunction/.style={
    subfunction,
    inner xsep=.2\masterunit,
    inner ysep=.15\masterunit,
    line width=.5\defaultpenwidth,
    /function/north io=0,
    /function/south io=0,
    /function/east io=1,
    /function/west io=1,
}}

\tikzset{partial/.style={
    draw,
    flow,
    minimum height=4\defaultpenwidth,
    minimum width=4\defaultpenwidth,
    strike out,
}}

\tikzset{
    brace flow/.default=0.2,
    brace flow/.code={
        \ifdim #1pt<0pt \tikzset{
            flow,
            decorate,
            decoration={
                brace,
                mirror,
                amplitude=.2\masterunit,
                raise=-#1\masterunit,
            }
        } \else \tikzset{
            flow,
            decorate,
            decoration={
                brace,
                amplitude=.2\masterunit,
                raise=#1\masterunit,
            }
        } \fi
    }
}

% distance, from, to
\newcommand\braceflow[3][.2]{
    \draw [brace flow=#1] #2 -- #3;
    \ifdim #1pt<0pt
        \coordinate (lastbrace) at ($ #2!.5!#3 + (0, -.2) + (0, #1) $);
    \else
        \coordinate (lastbrace) at ($ #2!.5!#3 + (0, .2) + (0, #1) $);
    \fi
}

\tikzset{measure width/.style={
    line width=0.02\masterunit,
}}

\tikzset{measure/.style={
    measure width,
    -{Triangle[angle'=30]},
}}

\newcommand\measure[4][.5]{
    \draw [measure width]
        (#3) -- ++(0, #1) coordinate (last measure left) -- +(0, .1) -- +(0, -.1)
        (#4) -- ++(0, #1) coordinate (last measure right) -- +(0, .1) -- +(0, -.1);
    \node (last measure) [outer sep=.1\masterunit] at ($ (last measure left)!.5!(last measure right) $) {#2};
    \draw [measure] (last measure) -- (last measure left);
    \draw [measure] (last measure) -- (last measure right);
}

\newcommand\bottomrightmeasure[3]{
    \draw [measure width]
        (#2) -- ++(0, -.6)
        (#3) -- ++(0, -.6);
    \coordinate (_m1) at ($ (#2) + (0, -.5) $);
    \coordinate (_m2) at ($ (#3) + (0, -.5) $);
    \node [outer sep=.1\masterunit, anchor=south west] (_mn) at (_m2) {#1};
    \draw [measure] ($ (_m1) + (-.5, 0) $) -- (_m1);
    \draw [measure] (_mn.south east) -- (_m2);
}

\tikzset{big arrow/.style={
  draw,
  line width=\defaultpenwidth,
  minimum height=.4\masterunit,
  minimum width=.3\masterunit,
  node contents={},
  single arrow,
  single arrow tip angle=60,
  single arrow head extend=.02\masterunit,
}}

\tikzset{<- iterate/.style={
    bend right=45,
    start angle=30,
    delta angle=60,
    x radius=.75,
    y radius=.5
}}

\tikzset{iterate ->/.style={
    bend left=45,
    delta angle=-60,
    start angle=150,
    x radius=.75,
    y radius=.5
}}

\tikzset{iterate up/.style={
    bend left=45,
    delta angle=-60,
    start angle=240,
    x radius=.5,
    y radius=.75
}}

\tikzset{iterate down/.style={
    bend right=45,
    delta angle=60,
    start angle=120,
    x radius=.5,
    y radius=.75
}}

\tikzset{next/.style={
    postaction={
        decorate,
        decoration={
            pre=moveto,
            pre length=.1\masterunit,
            raise=.1\masterunit,
            text along path,
            text=next,
            text align=left,
        },
        draw=none,
        tips=false,
    }
}}

\tikzset{next reversed/.style={
    postaction={
        decorate,
        decoration={
            pre=moveto,
            pre length=.1\masterunit,
            raise=.1\masterunit,
            text along path,
            text=next,
            text align=left,
            reverse path
        },
        draw=none,
        tips=false,
    }
}}

\tikzset{badge/.style={
    draw,
    circle,
    fill=white,
    minimum size=.4\masterunit
}}

\newcommand\true{\texttt{true}}
\newcommand\false{\texttt{false}}

\makeatletter
\newcommand\currentcoordinate{\the\tikz@lastxsaved,\the\tikz@lastysaved}
\makeatother

\newcommand{\drawbg}{
     \begin{scope}[on background layer]
         \coordinate (_a) at ($ (current bounding box.south west) + (-.5, -.5) $);
         \coordinate (_b) at ($ (current bounding box.north east) + (.5, .5) $);
         \fill [white] (_a) rectangle (_b);
     \end{scope}
}
