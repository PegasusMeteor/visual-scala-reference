\newcommand{\true}{\ifmmode $\textcolor{green}{\ding{51}}$ \else \textcolor{green}{\ding{51}} \fi}
\newcommand{\false}{\ifmmode $\textcolor{red}{\ding{55}}$ \else \textcolor{red}{\ding{55}} \fi}

\newlength{\cellwidth} \setlength{\cellwidth}{1.3cm}
\newlength{\cellheight} \setlength{\cellheight}{\cellwidth}
\newlength{\cellborderwidth} \setlength{\cellborderwidth}{0.4mm}
\newlength{\elementswidth} \setlength{\elementswidth}{1.4\cellwidth}

\tikzstyle{collection element}=[
  line width=\cellborderwidth,
  minimum width=\cellwidth,
  minimum height=\cellheight,
  outer sep=0mm,
]

\tikzstyle{collection}=[
  inner sep=0,
  column sep=-\cellborderwidth,
  row sep=-\cellborderwidth,
  text height=1.45ex,
  text depth=0.5ex,
  node distance=1mm,
  nodes={
    draw,
    collection element,
  }
]

\newcommand{\smallcolfactor}{.75}
\tikzstyle{small collection}=[
  collection,
  nodes={
    minimum width=\smallcolfactor\cellwidth,
    minimum height=\smallcolfactor\cellheight,
  },
]

\tikzstyle{empty collection}=[
  draw=gray,
  minimum height=\cellheight,
  dashed,
  line width=0.2mm,
  node contents={},
]

\tikzstyle{elements before}=[
  decorate,
  decoration={
    show path construction,
    lineto code={
      \tikzstyle{every path}=[line width=\cellborderwidth]
      \draw (\tikzinputsegmentfirst) [loosely dotted, dash pattern=on .2mm off 1mm] -- ($ (\tikzinputsegmentfirst)!.85!(\tikzinputsegmentlast) $);
      \draw ($ (\tikzinputsegmentfirst)!.85!(\tikzinputsegmentlast) $) -- (\tikzinputsegmentlast);
    }
  }
]

\tikzstyle{elements between}=[
  decorate,
  decoration={
    show path construction,
    lineto code={
      \draw [line width=\cellborderwidth]
        (\tikzinputsegmentfirst) -- ($ (\tikzinputsegmentfirst)!.15!(\tikzinputsegmentlast) $)
        ($ (\tikzinputsegmentfirst)!.85!(\tikzinputsegmentlast) $) -- (\tikzinputsegmentlast);
      \draw [loosely dotted, dash pattern=on .2mm off 1mm]
        ($ (\tikzinputsegmentfirst)!.15!(\tikzinputsegmentlast) $) -- ($ (\tikzinputsegmentfirst)!.85!(\tikzinputsegmentlast) $);
    }
  }
]

\tikzstyle{elements after}=[
  decorate,
  decoration={
    show path construction,
    lineto code={
      \draw [line width=\cellborderwidth] (\tikzinputsegmentfirst) -- ($ (\tikzinputsegmentfirst)!.15!(\tikzinputsegmentlast) $);
      \draw [loosely dotted, dash pattern=on .2mm off 1mm] ($ (\tikzinputsegmentfirst)!.15!(\tikzinputsegmentlast) $) -- (\tikzinputsegmentlast);
    }
  }
]

\newcommand{\elem}[3][] {
  \node (#2#3) [#1] {$#2_{#3}$};
}

\newcommand{\helem}[2][\elementswidth] {
  \node (x) [draw=none, minimum width=#1] {};
  \draw (x.north west) [elements #2] -- (x.north east);
  \draw (x.south west) [elements #2] -- (x.south east);
}
\newcommand{\velem}[2][\elementswidth]{
  \node (x) [draw=none, minimum height=#1] {};
  \draw (x.north west) [elements #2] -- (x.south west);
  \draw (x.north east) [elements #2] -- (x.south east);
}

\tikzstyle{map}=[
  collection,
]

\tikzstyle{maps to}=[
  decorate,
  decoration={
    show path construction,
    lineto code={
      \tikzstyle{every path}=[line width=.75\cellborderwidth]
      \fill (\tikzinputsegmentfirst) [white] circle [radius=2mm];
      \fill (\tikzinputsegmentfirst) circle [radius=1mm];
      \draw ($ (\tikzinputsegmentfirst) + (0, 2mm) $) [draw,  line width=\cellborderwidth] arc [start angle=90, end angle=270, radius=2mm];
      \draw (\tikzinputsegmentfirst) [-Triangle] -- (\tikzinputsegmentlast);
    }
  }
]

\tikzstyle{option}=[
  draw,
  line width=0.3mm,
  minimum size=1.1cm,
  outer sep=0
]

\tikzstyle{some}=[
  option,
  rectangle split,
  rectangle split parts=2,
  rectangle split draw splits=true
]

\newcommand{\some}[1] {
  \small \texttt{Some} \nodepart{two} #1
}

\tikzstyle{none}=[
  option,
  node contents={\small \texttt{None}},
]

\tikzstyle{tuple of collections}=[
  collection,
  inner sep=.25\cellheight,
  nodes={inner sep=0},
  left delimiter=(,
  right delimiter=),
]

\newcommand{\tuplecomma} {
  \node [draw=none, text height=\cellheight, minimum width=0.7cm, font=\Huge] {\,,\,\,\,};
}

\tikzstyle{function}=[
  draw,
  shape=function,
  line width=\cellborderwidth,
  minimum width=\cellwidth,
  minimum height=\cellheight,
  outer sep=0mm,
]

\tikzstyle{exception}=[
  draw,
  decorate,
  decoration={
    zigzag,
    segment length=2mm,
    amplitude=0.25mm
  },
  font=\ttfamily
]

\tikzstyle{arrow}=[
  draw=black,
  -latex,
]

\tikzstyle{reverse arrow}=[
  draw=black,
  latex-,
]

\tikzstyle{smooth}=[
  out=270,
  in=90,
]

\tikzstyle{smooth arrow}=[
  smooth,
  arrow,
]

\tikzstyle{arrow label}=[
  font=\footnotesize,
  fill=white,
]

\tikzset{horizontal bridge/.pic={
  \fill [white] circle [radius=1mm];
  \draw +(0, -1mm) -- +(0, 1mm);
  \draw +(-1mm, 0) arc [start angle=180, end angle=0, radius=1mm];
}}

\tikzset{vertical bridge/.pic={
  \fill [white] circle [radius=1mm];
  \draw +(-1mm, 0) -- +(1mm, 0);
  \draw +(0, -1mm) arc [start angle=270, end angle=90, radius=1mm];
}}

\newlength{\bigarrowwidth}
\setlength{\bigarrowwidth}{5mm}
\tikzstyle{big arrow}=[
  draw,
  line width=0.4mm,
  single arrow,
  single arrow tip angle=90,
  single arrow head extend=1mm,
  minimum width=\bigarrowwidth,
  minimum height=\bigarrowwidth,
  node contents={},
]

\tikzstyle{erase}=[
  preaction={
    draw=white,
    line cap=butt,
    -,
    line width=0.5em,
  },
  shorten <= 1pt,
  shorten >= 1pt,
]

\tikzstyle{start dots}=[
  decorate,
  decoration={
    border,
    angle=90,
    segment length=2pt,
    amplitude=\pgflinewidth,
    pre=curveto,
    pre length=0,
    post=curveto,
    post length=5mm
  }
]

\tikzstyle{middle dots}=[
  decorate,
  decoration={
    border,
    angle=90,
    segment length=2pt,
    amplitude=\pgflinewidth,
    pre=curveto,
    post=curveto,
    pre length=5mm,
    post length=5mm,
  }
]

\tikzstyle{iteration}=[
  arrow,
  dashed,
  gray!75,
]

\tikzstyle{with next label}=[
  postaction={
    decorate=true,
    decoration={
      text color=gray!75,
      text along path,
      raise=1mm,
      text={|\ttfamily\scriptsize|next},
      text align={align=center},
    }
  }
]

\tikzstyle{with reverse next label}=[
  preaction={
    decorate,
    decoration={
      raise=1mm,
      reverse path,
      text color=gray!75,
      text along path,
      text={|\ttfamily\scriptsize|next},
      text align={align=center},
    }
  }
]

\tikzstyle{plus step}=[
  postaction={
    decorate,
    decoration={
      raise=1mm,
      text along path,
      text align=center,
      text={|\ttfamily \scriptsize|+ step}
    }
  }
]

\tikzstyle{internal operation}=[
  text=black!75,
  scale=0.9,
]

\tikzstyle{predicate evaluation}=[
  node distance=.25,
  text=black!75,
  scale=0.8,
]

\tikzstyle{predicate evaluation edge}=[
  draw=black!50,
  line width=0.1mm,
  decorate,
  decoration={
    snake,
    amplitude=.5mm,
    segment length=2mm,
  }
]

\tikzstyle{inverse measure}=[
  arrows={Triangle[angle=15:3mm 1]-}
]
\tikzstyle{double measure}=[
  arrows={Triangle[angle=15:3mm 1]-Triangle[angle=15:3mm 1]}
]

\newcommand{\measure}[4][1ex] {
  \begin{scope}
    \tikzstyle{every path}=[line width=0.1mm]
    \draw (#3) -- ++(0, #1) +(0, 1.5mm) -- +(0, -1.5mm);
    \draw (#4) -- ++(0, #1) +(0, 1.5mm) -- +(0, -1.5mm);
    \draw ($ (#3) + (0, #1) $) [double measure] -- node [above] {#2} ($ (#4) + (0, #1) $);
  \end{scope}
}
\newcommand{\rightmeasure}[4][1ex] {
  \begin{scope}
    \tikzstyle{every path}=[line width=0.1mm]
    \draw (#3) -- ++(0, #1) +(0, 1.5mm) -- +(0, -1.5mm);
    \draw (#4) -- ++(0, #1) +(0, 1.5mm) -- +(0, -1.5mm);
    \draw (#3) ++(0, #1) [inverse measure] -- +(-6mm, 0);
    \draw (#4) ++(0, #1) [inverse measure] -- node [above, pos=.6] {#2} +(12mm, 0);
  \end{scope}
}

\tikzstyle{brace}=[
  decorate,
  decoration={
    brace,
    mirror,
    raise=2mm,
    amplitude=2mm,
  },
  line width=0.2mm,
]

\newcommand{\topbrace}[3][lastbracepoint] {
  \draw ([xshift=0.5mm] #2) [brace, decoration={mirror=false}] -- ([xshift=-0.5mm] #3);
  \coordinate (#1) at ($ ($ (#2)!.5!(#3) $) + (0, 4mm) $);
}

\newcommand{\bottombrace}[3][lastbracepoint] {
  \draw ([xshift=0.5mm] #2) [brace] -- ([xshift=-0.5mm] #3);
  \coordinate (#1) at ($ ($ (#2)!.5!(#3) $) - (0, 4mm) $);
}

\newcommand{\toppointer}[2]{
  \draw (#1) [latex-] -- ($ (#1) + (0, 1) $) node [above] {#2};
}

\newcommand{\param}[1] {
  \texttt{#1}
}

\newcommand{\subparam}[2] {
  \texttt{#1}_{\texttt{\scriptsize {#2}}}
}
 
\newcommand{\topparam}[2]{
  \node [arrow box, draw, arrow box arrows={south:2.5mm}, anchor=south] at ([yshift=1mm] #1) {#2};
}
\newcommand{\bottomparam}[2]{
  \node [arrow box, draw, arrow box arrows={north:2.5mm}, anchor=north] at ([yshift=-1mm] #1) {#2};
}
\newcommand{\leftparam}[2]{
  \node [arrow box, draw, arrow box arrows={east:2.5mm}, anchor=east] at ([xshift=-1mm] #1) {#2};
}
\newcommand{\rightparam}[2]{
  \node [arrow box, draw, arrow box arrows={west:2.5mm}, anchor=west] at ([xshift=1mm] #1) {#2};
}
